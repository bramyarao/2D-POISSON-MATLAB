\documentclass[9pt]{beamer}
% FOR HANDOUTS ONLY
%\documentclass[handout]{beamer}
%\usepackage{pgfpages}
%\pgfpagesuselayout{2 on 1}[a4paper,border shrink=5mm]
\usetheme{Madrid}
\usecolortheme{wolverine}
\usefonttheme{serif}
%\useinnertheme{rectangles}
\usepackage{verbatim} % for comments
\usepackage{mathtools}
\usepackage{ragged2e} % For justifying text
\newcommand\Myperm[2][^n]{\prescript{#1\mkern-2.5mu}{}P_{#2}} %Permutation
\newcommand\Mycomb[2][^n]{\prescript{#1\mkern-0.5mu}{}C_{#2}} %Combination


\title{2D Poissons}
\author{Ramya Rao Basava}
\institute{CS, UBC}
%\date{\today}
\date{} %no date


\begin{document}

%-----------------------------
% TITLE PAGE
%-----------------------------
\begin{frame}
\titlepage
\end{frame}

%-----------------------------
% CONTENTS PAGE
%-----------------------------
\begin{frame}
\label{contents}
\frametitle{Contents}
\tableofcontents
\end{frame}

%************************
\section{Problem Statement}

%-----------------------------
% Slide 1
%-----------------------------
\begin{frame}

\frametitle{Problem Statement}

Consider the 2D Poisson problem with essential boundary conditions as given
below:
\begin{align*}
\Delta u (x,y) &= (x^2+y^2)e^{xy} &\text{in} \quad \Omega = \{(x,y) \vert 0 < x < 1,0 < y < 1\} \\
u (x,y) &= e^{xy} \quad &\text{on }  \partial\Omega = \{(x,y) \vert x=0,1 \text{ and } y=0,1\}
\end{align*}

The analytical solution to the problem is $u (x,y) = e^{xy}$. 
\end{frame}


%-----------------------------
% Slide 2
%-----------------------------
\section{Numerical Analysis}
\begin{frame}

\frametitle{Numerical Analysis}
The domain is discretized using $10\times10$ source points and  $40\times40$ collocation points which do not overlap. The weight on the essential boundary is taken to be $\sqrt{\alpha^g}=N_S$, where $N_S$ is the number of source points. 

\vspace{8pt}
The numerical solution using RKCM obtained along the diagonal line passing through the points $(0,0)$ and $(1,1)$ is plotted in figure below and compared with the analytical solution. The RKCM result is very close to the analytical solution.

\end{frame}


\end{document}